\chapter{Architectural Overview}
\label{chapter:archi}
With a good understanding of basic concepts of the tehnology used we can go foward and
talk about the architectural of the entire system. I will talk separetly the two apps.


\section{Call App}
\label{sec:callapp}
\fig[height=0.5\textheight]{src/img/callApp.png}{img:call-app}{Call App}
Call App is the part of the system which main ocupation is the social interaction between
the robot users and their relatives seen as clients users, but you can also have this
between robots users or clients users. The big system is represented in \labelindexref{Figure}{img:call-app}.
Let's start talking about the entire workflow of calling, but firstly
we will use some notation: Alice for the client or robot that is starting the call and Bob for the client or robot reciving the
call. If Alice wants to call Bob she cand do that only if he is online. If Bob is online than Alice send a message
different from normal messages telling Bob that she wants to start a call. Now Bob can refuse the call and send back to
Alice a reject message and then both stops for having a call or he can accept the call and send back to Alice a message
telling that he accept the call. When Alice recive the message she create a RTC offer package (that include some
basic networking information about Alice and maybe some ICE candidates) and send it to Bob, in the same she set 
her local description with this offer description and she starts to gather ICE candidates and send them to 
Bob when she finds them. Bob recive the RTC offer , set his remote description with this offer description, and creates an
RTC answer that has almost the same information as the offer, set this answer as his local description
and send it to Alice, in the same time he starts to gather ICE candidates and send them to Alice.
Alice recive the RTC answer and set it as remote description. After some ICE candidates changes between Alice and
Bob if they do not find any matches for a RTC connection they will closed it. If they found they will go to connection
established state and they will start to transmit audio/video data. They are tree types of connection posible.
First one if Alice and Bob are in the same network or both have public ips they cand make a direct connection (r-c.1).
Second one if Alice or Bob are under a different NAT network they can make a direct connection with the help of a STUN Server
or they comunicate through a common TURN server (r-c.2 adn r-c.3). Third one if Alice and Bob are under a different NAT
network and they do not have a common TURN server accesible but the servers are accesible between them they cand make
a connection trought two servers (r-c.4). At one moment if Alice or Bob wants to end the call they send a stop message, After
both close the connection. A perfect workflow is represented in \labelindexref{Figure}{img:call-AB1}.
\fig[height=0.5\textheight]{src/img/ABCall.png}{img:call-AB1}{Alice-Bob call}

\subsection{Robots}
\label{sub-sec:callApp-robots}

The robots have the webpage application implemented in the hmi_bridge ROS package that is part of
the big project ENRICHME, so it does need any connection to frontend server for taking the html or
JavaScript scripts. It only makes a connection using SOCKET.IO to the backend server called in the 
\labelindexref{Figure}{img:call-app} r.1. This connection authentificate the robot to the
server and announce the others users that is online and it can have a video call. Trough this connection
the robots can recive events like when a user go online or offline, recive its contacts list, recive
particulary information about some users. Some other two importants thinks that goes trough this connection
are the messages and the informaton about starting a call. First the messages, the robot can send them trough
r.1, if the other client or robot is online it will recive the message trough its r.1 or c.3 connection, if is
offline than the message will be keep in the MongoDB database on the backend server (or in some specialised MongoDB
database servers) and the recive users can request them lately trough REST API or SOCKET.IO connection. All the messages
are keep in database, so the users cand see them lately. For the chat part the robots see their ROS parametre fro language
and makes the right keyboard for the user. Secondly the call part, the robots use their r.1 for transmmiting 
a call request or to respond a reciving call, and for all the other messages related to the call app, except
audio/video data. You cand send audio/video over the r.1 connection but will have a bad performance because it
used TCP not UDP and also it always go trough the server. Some difference between robots and normal clients are
that robots do not used the camera so the video is taken trough a ROS topic for a RGB camera and you can not add
connection on the robot because of the type of users that we have on this aprt of application.

\subsection{Backend Server}
\label{sub-sec:callApp-backend}
The Backend Server offer a REST API and SOCKET.IO connections. The REST API offer services like
creating a new user, adding new contacts to the contacts list of an user, basics information about
a user, authentification for users, old messages between users, contacts list. The SOCKET.IO connections
are used for sending normal messages between users or special messages needed to make RTC connection
between users and to keep an online/offline track of the users. In my implementation the MongoDB
database is also on the backend server, but if the system can be implemented with the database
on another server or on multiple servers agregate for more stability and failure response. Because of
the online/offline feature implemented in SOCKET.IO this system is limited at only one backend sever.
Changing this feature in MongoDB database but making more a pooling from the clients and robots will
incrase the scability of the backend server, but also the will incrase the bandwidth used and
decrease performance when the system has less number of users than 40000. Because this number will
not be reach in the project ENRICHME I used this type of architectural that makes the implementation
more easy to understand and developing. The backend server run on two ports 80 and 443. I will discus
the details in the next subsections.


\subsection{Clients and Frontend Server}
\label{sub-sec:callApp-clients}
Because only clients use the frontend server I chose to talk about them togheter. The clients make
a request to the frontend server for the html and JavaScript scripts (c.1) and get a them as
response (c.2) and after that they make the SOCKET.IO connection with the backend server (c.3).
I made the design of the page in only one html file and in more JavaScript scripts so the users
will have the sensation of using an app not a webpage, and also for the performance, because after
the first request the clients will not request anything more from the frontend server. This design and
the fact that the project does not have a big number users the performance is high. If there it need to
implement the system with less cost we can fusion the frontend server and the backend server without 
cost to performance too high. Because browsers let webcam and microphone share only over secure connections
or locahost inscure connection (more for testing) it is a must to have a ssl certificate for the
frontend servers. Because of same origin policy and security issues for browsers also the backend
server need a certificate and that is why the backend server is running on two ports 80 for the robots
that are using an locahost webpage and 443 for the clients.


\subsection{STUN/TURN servers}
\label{sub-sec:callApp-stun-turn}
STUN/TURN servers are used for making connections between users that transmit video/audion
data. Because of some policy of firewalls or proxy is best to have more of this types of servers.
I used on made by me on the amazon coud system and a public one, but free, server from numb.viagenie.ca.
The role of STUN is that the clients can discover their network information about them if they are
under a NAT network. The TURN servers come in used when users can not make a direct connection and they
react as a relay server for their data.

\section{News App}
\label{sec:newsapp}
\fig[height=0.5\textheight]{src/img/newsApp.png}{img:news-app}{News App}
\labelindexref{Figure}{img:news-app}.
The news app is a news agregator which takes news from different newspapers websites and
show them to the user in a nice format, with the option to be read by the robot. Because
it is not posible anymore to take an rss feed with the same origin policy implemented in
browsers I used an internet API call rss2json.com that takes rss feed you want and transform
it in a JSONP format that can bypass this problem. The architecture is simple for this app.
You have the newspapers web sites that are publishing rss feed, that also have different
feeds for different categories. The rss2json.com API server, where you send your API KEY
with the rss feed url and the number of news, is your providers of JSONP formated news.
The robots recive the JSONP and show the news to the users with the option to be read loud
by them. Users also have the option to change the categories of news they want and even the
newspapers.
