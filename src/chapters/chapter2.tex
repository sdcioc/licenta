\chapter{Basic Concepts}
\label{chapter:basic}
In the previous chapter I presented a short introduction of the ENRICHME project and subjects of work.
Before I present a more detailed description of the main goals and the implementation of them I would like
to discuss about the basic conecepts of working with robots and of the tehnologies and programing languages used.


\section{Theoretical Background}
\label{sec:tbkg}

To begin with, I will make a description of the theory behind my work. This includes
presenting the robot, the protocols used for real time comunication, taking news from the news sites
and some security basic browser security.

\subsection{Heart Rate}
\label{sub-sec:tbkg-heartrate}



\subsection{WebRTC}
\label{sub-sec:tbkg-webRTC}
WebRTC (Web Real-Time Communication) \cite{bergkvist2012webrtc} is a open-source project
that does not need plugins to be used on browsers. It has a collection of communications
protocols and APIs (application programming interfaces) that enable real-time
communication between peers.
That are two distinct APIs C++ API (fore browser developers) and Web API
(for web developers, javascript API).
I used the second one because is more portable
between diferent types of systems.
Even if WebRTC allow you to do file transfer, chat or desktop sharing 
in my project is necesarly only the video coferencing feature used.
For this to work you need a STUN server \cite{mahy2003network} and 
for some networks a TURN server \cite{mahy2010traversal}.
The tehnology used for creating the connexion peer-to-peer is represented
by the ICE \cite{rosenberg2010interactive}.
For transmiting the audio and video data webRTC use RTP (Real-Time Transport Portocol).


\subsection{RSS 2.0 Feed}
\label{sub-sec:tbkg-rss}
RSS 2.0 (Really Simple Syndication) \cite{board2014rss} is know as a tpe of web feed \cite{downing2008web}.
A web feed (news feed) is a data format to provide informations which frequently change their contents.
It create a plublisher-subscribers system.The publisher create a web feed and the users subscribe to it.
Mostly the rss feed is used for news media to send micro-news to their customers. with this tehnology you can
create a news agregator where you have multiple rss feeds from different sites and you filter them for your users.
They can see a short description to the news, and after that they can have the option to go to the entire news
on the parent site. \\
There are two types of web feed most used: rss feed and atom feed. I used rss feed because is implemented by most
of the news sites and I wanted a big base of news. Because of the browsers same origin policy (\labelindexref{Section}{sub-sec:tbkg-sameorigin})
 I had to use an API that transform rss feed in jsonp.


\subsection{Same Origin Policy}
\label{sub-sec:tbkg-sameorigin}
Same Origin Policy \cite{network10same} is a concept used in web application security model. In browsers which respect this policy
you are not allowed to use data in scripts from pages that do not have the same origin. This policy must be respect so you will not have
malicious code running in your browser or your webpage. Origin is defined as concatanation of
the protocol used, hostname and port of a website. You can bypass this restriction by using the cross-origin communication
with JSONP \cite{ozses2009cross}.

% ROS section
\section{ROS}
\label{sec:ros}

\subsection{ROS-Topics}
\label{sub-sec:ros-topics}
\subsection{ROS-Publishers}
\label{sub-sec:ros-publisher}
\subsection{ROS-Subscribers}
\label{sub-sec:ros-subscribers}


\section{JavaScript}
\label{sec:javascript}
"JavaScript is an interpreted programing language with object-oriented (OO)
capabilities. Syntactically, the core JavaScript language resembles C, C++,
and Java, with programing constructs such as the if statement, the while loop,
and the \&\& operator. The similary end with this syntactic resemblance, however.
Javascript is a loosely typed language, which means that variables do not need
to have a type specified. Objects in JavaScript map property names to arbitrary property
values. In this way, they are more like hash tables or associative arrays(in Perl) than
they are like structs (in C) or objects (in C++ or Java). The OO inheritance mechanism
og JavaScript is prototype-based like of the little-known language Self. This is quite
different from inheritance in C++ abd Java. Like Perl, JavaScript is an interpreted
language, and it draws inspiration from Perl in a number of areas, such as its regular-expression
and array-handling features" \cite{flanagan2006javascript} \\
I used JavaScript in my project as the frontend language , but also as backend of the authentication and
signal sever. JavaScript is an easy to use programing language and it has more features and developing
than php. Another feature is that is more client-side than php. In the next subsections I will present
some framework used in my developing.


\subsection{JQuery}
\label{sub-sec:javascript-jquery}
JQuery is a JavaScript library which make thinks like HTML (Hyper Text Makeup Language)
document traversal and manipulation, event handling, animation, and Ajax more easy-to-use.
It is compatible with most of browsers and also it has behind it a big comunity. I used
Jquery because it make the developing of websites fast and easy to understand by outher
developers, but some parts of the code was done in pure JavaScript.


\subsection{SOCKET.IO}
\label{sub-sec:javascript-socket}
SOCKET.IO is a JavaScript library (frontend and backend APIs) used for making
realiable real-time web application. You used it for making a connexion beetween
the webpage given by the webserver to the backend server, mostly for message from
backend server. It can use two methods. The first one used websockets that makes
things fast and friendly with network bandwidth. Second it used pooling that is the
clasic way to ask the server from time to time if it has something to send to you.
\subsection{Sweet Alert 2}
\label{sub-sec:javascript-swalert}
Sweet Alert 2 is JavaScript library and also a css file, which make the alerts from
the browser customizable. We need this library because the application is mostly
used by old people we need to have alerts that have bigger text and are more center
in the webpage. Also for the call part you cand have multiple option for responding
the alert.


\subsection{Node.js}
\label{sub-sec:javascript-node}
Node.js is open-source JavaScript run-time environment that allow to write server-side
JavaScript code and it is saw as on of the fundamental elements of "JavaScript everywere"
paradigm. This make the client-side code and server-side code to be write in the same
programing language that makes it more easy-to-understand for both parts. I used it
for the server-side of my web application. I used also some extra library with it like
SOSCKET.IO, JWTOKEN and some paradimgs like REST. Node.js is a fast as performance and
also as time of developing, and because is JavaScript there is easy for other developers
to use.


\subsection{Roslibjs}
\label{sub-sec:javascript-roslib}
Roslibjs is the standard JavaScript library for interacting with ROS systems from
a browser. It connects with rosbridge trough WebSockets and has an API that cand let
you do things like publishing, subscribing, calling services and most of the essential ROS
functionality. Because a big part of my project is the web application in a browser which 
run on a system with ROS I had to use as my only way to comunicate with ROS and other parts
of the big project

\section{Python}
\label{sec:python}
"
Python is a general-purpose, high-level programming language whose design philosophy emphasizes code readability.
Python's syntax allows programmers to express concepts in fewer lines of code than would be possible in languages
such as C, and the language provides constructs intended to enable clear programs on both a small and large
scale.
Python supports multiple programming paradigms, including object-oriented, imperative and functional programming
styles. It features a fully dynamic type system and automatic memory management, similar to that of Scheme, Ruby,
Perl and Tclm and has a large and comprehensive standard library.
Like other dynamic languages, Python is often used as a scripting language, but is also used in a wide range of
non-scripting contexts. Using third-party tools, Python code can be packaged into standalone executable programs.
Python interpreters are available for many operating systems.

CPython, the reference implementation of Python, is free and open source software and has a community-based
development model, as do nearly all of its alternative implementations. CPython is managed by the non-profit Python
Software Foundation.
" \cite{van2007python}
\subsection{Rospy}
\label{sub-sec:python-rospy}
Rospy is a Python client library that offer an API for interacting with ROS systems.
It allows you to do interface with topics, services and parameters. The design of rospy
is very easy for programers to use and take a less time in developing. This makes rospy
a very good aproach for prototyping in ROS. Some of the ROS tools are made with rospy.


\section{MongoDB}
\label{sec:mongo}
MongoDB \cite{chodorow2013mongodb} is a document-oriented and general-purpose database. I used it because
its documents are in fact BSONs (similar to JSONs) which are very easy to use in JavaScript and Python.
Also MongoDB is know to be a scalable database, so it makes easy implementation of
the chat part, where scalability is a must. On the server-side I used it for keeping
the user credentials, informations, chat messages and connections list, on the robot-side
for keeping the news preferences in newspapers and news categories of the user. Because I use Node.js,
but also Python, I had to use some libraries, that I will present next.
\subsection{PyMongo}
\label{sub-sec:mongo-py}
PyMongo is a Python distribution that make a connexion between Pyhton and MongoDB. It is easy to use by
developers and easy to understand. Also you can transform data recived from the database in dictionaries
in Pyhton.
\subsection{MongoDB Node.js Driver}
\label{sub-sec:mongo-node}
For Node.js we have an official MongoDB driver that provides both callback and promise interaction.
This driver is easy-to-use and easy-to-understand by all MongoDB users, because it use almost the
the same interface.
