\chapter{The News Module adaptation experiment}
\label{chapter:experiment}
\section{Experiment design and data analysis}
\paragraph{TASK:}
This experiment test the adaptation of the News Module. The adaptation consists in the changing of the font size of the news. 

\paragraph{SENSORS:}
In this experiment we are going to record the RGB, the thermal data, the GSR and the heart rate.
From the RGB we are going to extract the AUs, and the blinking
From the thermal data we are going to extract the temperature on the face of the participants
We are going to extract the HEART RATE by using the physical sensor (it use the HRBSM)
We are going to extract the GSR by using the physical sensor

\paragraph{CONDITIONS:}
It is going to have 6 conditions:
\begin{enumerate}
\item: Newspaper is read aloud by the robot (no interface shown)
\item: Newspaper is shown on the torso of the robot (no sound) and no font size adaptation
\item: Newspaper is shown on the torso of the robot (no sound) and the font size is adapted
\item: Sound and visual interface with no font size adaptation
\item: Sound and visual interface with font size adaptation
\item: 5 with moving from 2m to 60cm
\end{enumerate}
This means 16 interactions per participant.


\paragraph{QUESTIONNAIRES:}
It is going to use the following questionnaires:
\begin{itemize}
\item MEQ -> to see if morningness/eveningness has an effect
\item AASP -> to see if a person prefers visual or auditory
\item EPQ -> to see if personality influences the distance where the person feels most comfortable
\end{itemize}

\paragraph{SCENARIO:}
The experiment has the following scenario:
It use an armchair and a Tiago robot.
The robot will invite the person to take a seat on the armchair, it will give the instructions and then the experiment can begin.
The robot will show/read the news at three different distances from the person:
\begin{itemize}
\item 2 m -> close phase of social distance
\item 1.2 m -> far phase of personal distance
\item 60 cm -> close phase of personal distance
\end{itemize}
At each distance the robot will display/read one news, once it finishes reading or the participant confirms that he/she finished reading the news it will move to the next distance.
Once all the distances have been covered, the robot informs the participant that the experiment is over and he/she can leave the room.


The idea of the experiment is to find out which is the best combination (distance, font size, speech/no speech) based on the sensory profile of the participant,
its personality, and current internal state. 

Each participant will test all conditions. In the end, we will need around 15-20 participants.

\section{Thesis experiment implementation}
\fig[height=0.4\textheight]{src/img/experiment.png}{img:experiment}{Experiment architecture}
For the experiment the News Module has some difference:
\begin{enumerate}
    \item three new css files for every one of the three distances
    \item getParams - subscribe to ROS topic /distance and every time the distance change
    it verify in what range it is and loads the necessary css file. This adapt the interface to
    the distance between user and robot.
\end{enumerate}
The HRBSM run on a ROS package. The heart rate sensor is place on the person.
And the HRBSM use its specific address fro connectiong to it. With the help of the
others module we take data like blinking, GSR, temperature of the face, and AUs.

More experiemtns are currentyly done this will be submited on ICSR in one week.

