\chapter{User Interface}
\label{chapter:user-interface}
With the knwoledge of architectural we can go foward to the user interface. This is also
an important part of the project because must adapt for persons with speacials needs.
That is the reason why on robots you have buttons and the font-size bigger than client
side of the page.


\section{Social Module User Interface}
\label{sec:ui-callapp}
The SM is separated in three parts. First part is the home which contains a list
of you connections. On client you have also the option to add new connections. If you
press any of the users names in the list it will pop-up a small window that contains
information about the user and options like :

\begin{itemize}
  \item Message - change page to chat part
  \item Call - change page to video-conference part
  \item Ok - close the pop-up
\end{itemize}

\paragraph{Chat}
\label{sub-sec:ui-callapp-message} 
\fig[height=0.2\textheight]{src/img/messageRobot.png}{img:message-robot}{Chat UI robot}
\fig[height=0.2\textheight]{src/img/messageui.png}{img:message-client}{Chat UI client}
The chat part on robot has a keyboard in a specific language(almost half of the screen),
a box that contains the messages between users and three specific buttons:
\begin{itemize}
  \item First button - give you more information about the remote user
  \item Get past messages - get messages from the past conversations
  \item Go to home - change page to home page
\end{itemize}
The client side does not hava keyborad, therefore it has more space for messages.

\paragraph{Video-conference}
\label{sub-sec:ui-callapp-call} 
\fig[height=0.2\textheight]{src/img/callRobot.png}{img:call-robot}{Call UI robot}
\fig[height=0.2\textheight]{src/img/callClient.png}{img:call-client}{Call UI client}
The video-conference part is almost the same on robot and client (the robot has the extra menu button on top).
You have on top three buttons:
\begin{itemize}
  \item First button - provide information about the remote user
  \item Home - change page to home page
  \item Message - change page to chat part with the current remote user
\end{itemize}
Under them there are three more buttons specific to call. The three buttons are:
\begin{itemize}
  \item Call - start a call with the remote user
  \item Accept - accept a receiving call
  \item Hangup - reject a receiving call or hang-up the current call
\end{itemize}
First, only the call button is available. If you call someone it will be diabled and
the hangup button will be activated. If your call is accepted nothing will happen to avability
of buttons. If your call is rejected then a pop-up will apear saying that you call was rejected
by the remote user and the avability of buttons it will be the one on the start. If you are reciving
a call then the call button will go disabled and will have availabled the accept and hangup button.
Under this buttons are your local video (your image from the camera, on the right) and remote video
(remote user image from their camera, on the left)

\section{News Module}
\label{sec:ui-newsapp}
\fig[height=0.2\textheight]{src/img/newsui.png}{img:news-ui}{News UI news}
\fig[height=0.2\textheight]{src/img/prefferences.png}{img:prefferences-ui}{News UI prefferences}
The NM has two parts: the first part where the news are showed to the user and and the second one
where the user can change his prefferences about newspapars and categories of news. The News Module is
presented only on robot side.

\paragraph{News}
\label{sub-sec:ui-newsapp-news}
The news part has four buttons on top and the news content on the
rest of the page. The buttons are: 
\begin{itemize}
  \item Previous button - change the current news content with the previous one content
  \item Next button - change the current news content with the next one content
  \item Read news button - makes the robot read your news content description
  \item Prefferences button - change your page to prefferences where you can chose newspapers and
  categories
\end{itemize}

\paragraph{Prefferences}
\label{sub-sec:ui-newsapp-prefferences}
The prefferences page has:
\begin{itemize}
  \item A list with newspapers that you can select because they are buttons. They have two colors
  green for selected and yellow for unselected
  \item A list with categories that you can select because they are buttons. They have two colors
  green for selected and yellow for unselected
  \item Back button - change your page with news page
\end{itemize}