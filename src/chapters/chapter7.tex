\chapter{Conclusion and future work}
\label{chapter:conclusion}
The thesis implement three modules on the project ENRICHME which are currently used.
Over this it provides three frameworks which can be used by developers in other
project making the implementation easier. The experiment is also an framework
for testing the adaptation of your application.

The three JavaScript frameworks are:
\begin{enumerate}
    \item lic_socket_object - a signaling API for comunicating with the backend server
    using a countinous connection. This is an alternative for REST API, in case you want
    an app more interactiv
    \item lic_call_object - a video calling API over webRTC which does not contains
    the hard to understand architectural of webRTC. This handle inside the errors where
    the problems which can raise when a peer-to-peer connection is created. Also the API
    has few command and very easy to understand for futher developing
    \item lic_news_object - a news agregator API which pass the security problems of rss
    and transform them into JSON format easier for developers to use without any need
    of conversions.
\end{enumerate}

The implementation of this modules over ROS makes incrase their portability and open
a field on using WebRTC protocol on robots.

\paragraph{Future Works}
The future works for the modules:
\begin{itemize}
\item Social Module will be to implement the WebRTC for ROS on C++ and Python, and creating an TURN server which works over proxy networks. 
\item News Module will be to implement an stand-alone open-source rss2json.com API, and doing some adaptaion for the categories of the news or newspaper.
\item Heart Rate Bluetooth Sensor Module transform it into a more generic module, not only for heart rate, and not only for BLE device
\item The experiment to be run also for the other parts of the ENRICHME project.
\end{itemize}