\chapter{Introduction}
\label{chapter:introduction}
%"
%Human-robot interaction (HRI) for socially assistive
%robotics has received much attention in recent years.
%Researchers are trying to develop autonomous systems that
%are capable of natural interaction with humans, that can learn
%from their past experiences, and can adapt their behavior
%based on the reaction of the person they interact with.
%All this is done so that robots can be accepted as companions
%for people that need special care (e.g., individuals with
%cognitive and/or physical impairments). The robots can
%provide monitoring, and help, they can alert
%the family in case of emergencies, and never get tired.


%This thesis is part of the Horizon2020 ENRICHME
%European project, whose purpose is to develop a socially
%assistive robot for elderly with mild cognitive impairment
%(MCI) to help them remain active and independent for longer
%and to enhance their quality of life. The project has partners
%from Italy, United Kingdom, France, Netherlands, Poland,
%and Greece. The system is going to be tested in Ambient
%Assisted Living (AAL) laboratories and in elderly housing.
%The AAL laboratories are Fondazione Don Carlo Gnocchi,
%in Milano, Italy, and Stichting Smart Homes in the
%Netherlands. In both facilities 15 older people will take part
%in the testing phase. The elderly caring facilities are Lace
%Housing Ltd, in the United Kingdom, Osrodek Pomocy
%Spolecznej in Poland, and Aktion Licenced Elderly Care
%Units in Greece. In each of the three sites 9 elderly will
%interact with the robot.
%The people that can participate in testings need to have
%over 65 years old, they have to been given the diagnostic of
%MCI by a neuropsychologist or to be identified with mild
%cognitive dementia on the MME score. Another aspect
%considered is the independence in executing basic activities
%for daily living assessed through Barthel ADL index.
%And lastly, the people need to be living in one of the studied
%facilities.

%" \cite{agrigoroaie2016developing}

%In the last few years, a large number of research has been made in order to develop a framework for a behavior
%control architecture that can provide customised interaction between a robot and a human being. Researchers are
%more and more focused on human-robot interaction. The main goal of research is to create independent
%systems that are capable of natural interaction with humans, that are able to understand human’s emotions and personality,
%and can behave appropriately in a social context. Concerning the elderly people and people who need special care,
%robots can become helpful companions that can provide monitoring and assistance. 
%Researchers have been develop different systems for people needing physical, emotional and cognitive assistance.
%A group of researchers presented a new system based on the socially assistive robotics that aims to provide a
%customized help protocol through motivation, encouragements, and companionship to users suffering from cognitive
%impairment using music. However, few researchers concentrated their attention on developing system that can adapt
%to human’s profile. The project ENRICHME aims to develop a behavioural control architecture that episodic memory,
%learning, and adaptation to the social context. The project has partners from Italy, United Kingdom, France, Netherlands,
%Poland, and Greece. The AAL laboratories are Fondazione Don Carlo Gnocchi, in Milano, Italy, and Stichting Smart Homes in the Netherlands. 
%The testings are designed for people over 65 years old suffering for mild cognitive impairment. Another criteria is e independence in
%executing basic activities for daily living assessed through Barthel ADL index. 
%Taking into consideration, the feedback given by by the elderly and their families and caregivers, specific needs have been identified
%such as calling somebody in case of an emergency, reminding elderly about the medication, monitoring the environment and suggesting improvements,
%monitor health state etc. The project propose a system that is able to provide personalised interaction with the person and adopt its behaviour,
%the robot planning its actions based on the personality and needs of the person. \cite{agrigoroaie2016developing}

This work is part of the Horizon2020 ENRICHME European
project, whose purpose is to develop a socially assistive
robot for elderly with mild cognitive impairment (MCI) to
help them remain active and independent for longer and
to enhance their quality of life. The project has partners
from Italy, United Kingdom, France, Netherlands, Poland,
and Greece. The system is going to be tested in Ambient
Assisted Living (AAL) laboratories and in elderly housing.
The AAL laboratories are Fondazione Don Carlo Gnocchi, in
Milano, Italy, and Stichting Smart Homes in the Netherlands.
In both facilities 15 older people will take part in the testing
phase. The elderly caring facilities are Lace Housing Ltd, in
the United Kingdom, Osrodek Pomocy Spolecznej in Poland,
and Aktion Licenced Elderly Care Units in Greece. In each
of the three sites 9 elderly will interact with the robot. The
robot chosen for this project is the Tiago
robot developed
by PalRobotics. In each of the testing centers there is going
to be one robot. The system is going to be simultaneously
tested in all the testing sites. The people that can participate
in testings need to have over 65 years old, they have to
receive the diagnostic of MCI by a neuropsychologist or to
be identfied with mild cognitive dementia on the MMSE
score. Another aspect considered is the independence in
executing basic activities for daily living assessed through
Barthel ADL index. And lastly, the people need to be living
in one of the studied facilities.
“ENRICHME” refers to the goal of enriching the day-to-day experiences of 
elderly people at home by means of technologies that enable health monitoring, complementary care and
social support, helping them to remain active and independent for longer and to enhance their quality of life. 
This work is focused on the time-extended personalized interaction while remaining in its own home
environment.  The only comparable technology is computer-based interaction, which could provide
personalized care but without the physical embodiment of a robot. The experiments proposed in this work
address the key differences between robot-based and computer-based interaction. 
\cite{agrigoroaie2016developing}



\section{Thesis description and objectifs}
This thesis contains the description and implementation of the three modules inside the ENRICHME project.
Two of them are graphical user interface module which are used in the web application presented on the
Thiago robot (the robot of the project). The other one is ROS module which takes data from an hardware
sensor (Heart Rate Sensor) and publish it on a ROS topic so it can be accesible by other modules and used.

For the first two the objectifs are:
\begin{itemize}
	\item User interface that can be understand even if the user have some issues of seing.
	\item An intuitive and simple aplication that can be used first time without training.
	\item Portable Aplication - can be used on multiple robots.
	\item Adaptive Aplication - it adapts after the user.
\end{itemize}

For the ROS module the objectifs are:
\begin{itemize}
	\item Real-time - the module must work in real-time.
	\item Interacting with other modules - it can interact with the other ROS modules.
	\item Error handling - it handles some errors unrealistic
\end{itemize}

Before the implementation of this modules the ENRICHME project did not have many interaction with things
outside the AAL laboratories and elderly housing. The Social Module allow 
users to interact between them with chat messaging or video-conference calls. The News module takes news from
internet sites and provides their content to the user. This modules is a step foward for introducing data
outside the AAL and see their affects on the users. For the News Module the thesis contains an experiment
on its adaptation and effects on humans.
All the interface modules must be compatible in all the language used in the project.


\section{ENRICHME project}
Improving the quality of life of the elderly and the less able people has been for a long time one of the main challenges for
the professionals in robotics. But in the last few years, technological progress and the development of common software platforms
started to have a great positive impact in this field. The European project ENRICHME which aims to better monitor and prolong the
independent living of the old people has developed an integrated system that can substantially help at achieving this goal. 
The project develops new technologies that enable health monitoring, complementary care and social support, helping elderly to
remain active and independent for longer and to enhance their quality of life. More specific, the project tries to consolidate
mobile robots within a smart-home environment o provide new Active and Assisted Living (AAL) services for the older person.
Moreover, the project tries to combine ambient intelligence (smart home sensors and services), robot intelligence(HRI, robot sensors and services)
and social intelligence (networked care, medical interface). ENRICHME system uses technological solutions and it mostly focuses on the integration
of smart-home and robot sensors for human and object localisation, activity monitoring and detection of abnormal behaviours. The forth most important
aspects of the project are:
\begin{enumerate}
	\item Distributed architecture for Ambient Assisted Living (AAL)  services based on ROS2; 
	\item Probabilistic solution for object localisation with a mobile robot based on RFID technology 
	\item Vision-based approach for estimating the level of activity of a person;
	\item Entropy-based system for detecting anomalous motion patterns at home;
\end{enumerate}

\section{Related work }
\paragraph{ENRICHME}
In the following lines I will try to present the work related with the ENRICHME program focusing first on the state of the art on AAL,
and second on the Radio Frequency Identification. Ambient Assisted Living aims to provide new technologies for helping people to remain
independent, based on ambient intelligence. The goal is to create infrastructures for life scenarios. For instance, the ALL architecture
introduces the CASAS “smart home in a box” which is a smart home design that is easy to install and provides smart home capabilities out
of the box with no customisation or training. \cite{cook2013casas} This allows smart home technologies to achieve its benefits at a large scale. CASAS has
three main conceptual levels. First, the physical layer and the lowest level contains hardware components including sensors and actuators
and its architecture utilizes a Zigbee wireless mesh which communicates directly with the hardware components. Second, the middleware layer 
is governed by a publish/subscribe manager. Moreover, this paradigm acts as a memory system, the manager providing named broadcast channels
that allow component bridges to publish and receive messages. In the end, an application layer lies on the top which allows control to move
down from the application layer to the physical components that automate the action. In Europe, one of the most important framework is UniversAAL
an open platform intended to facilitate the development, distribution, and deployment of technological 
solutions for ALL. It consists of an extensive set of resources (some are software and some are models/architectures) aimed to benefit end users
(i.e., assisted persons, their families, and caregivers), authorities with responsibility for AAL, and organizations involved in the development
and deployment of AAL services. The resources are classified into three main groups: runtime support, development support, and community support.
UniversAAL offers a semantic and distributed software platform designed to ease the development of integrated AAL applications. It makes use of
a complementary middle complementary middleware, called open- HAB3, to control various domestic sensors. \cite{ferro2015universaal}
Radio Frequency Identification (RFID) are a common and useful tool in recent AAL systems. These tools are used to locate
a human or an object within indoor environments. However, this is only possible by caring a small, inexpensive tag.
RFID localization is an interesting technology that uses classical techniques such as Time of Arrival, Time Difference 
of Arrival, Received Signal Phase or Trilateration, based on radio-frequency propagation models, including physical aspects of RFID communication. \cite{bouchard2014human} 
There are various approaches for locating an object in an environment. Huiting and other professionals propose statistical models of a tag detection event itself. \cite{huiting2013exploiting} 
Moreover, Soltani and Motamed propose proximity-based methods of creating, sharing, exchanging and managing the building information throughout the lifecycle among
all stakeholders. \cite{soltani2013neighborhood}

Even though a lot of methods have been proposed, RFID technology is more helpful when it is necessary
to monitor an activity. Most of the research have concluded that accelerometers have been widely considered as practical sensors
for wearable devices to measure and assess physical activity of people. Ravi and others concluded that activities can be recognized with
fairly high accuracy using a single triaxial accelerometer. Activities that are limited to the movement of just hands or mouth (e.g brushing)
are comparatively harder to recognize using a single accelerometer worn near the pelvic region. \cite{ravi2005activity} Atallah uses a light- weight earworn accelerometer
to categorize the daily activities of people into four levels: very low (e.g., sitting), low (e.g., reading), medium (e.g., preparing food), and iv)
high (e.g., sports). However, using video-based approaches are less intrusive and efficient solutions to analyse human activity. The market offers
also some benefits for video based approaches since we have observed a decrease in the prices for video cameras and there is a growing interest of
video-based methods to provide a smart environment for the elderly. Pal and Abhayaratne propose a framework with two different variants of the motion
features captured from two camera angles and classifies them into different activity levels using neural networks.  \cite{pal2015video} Optical flow vectors are used to detect
regions on the image that represents a motion. Furthermore histograms of oriented gradients features are extracted in these regions and finally,
a neural network is trained to classify the level of activity. However, this came with one inconvenient the noise of the optical flow, and thus,
extracting histograms of oriented gradients features can be expensive. The ENRICHME project has a different approach detecting the activity level
of the person by analysing the global and local motion of the whole body and of some body parts. 
It is also very important to assess the overall motion behaviour of the person over extended period of times.
In some cases, the detection of anomalous activity levels could indicate different health conditions. 
Different states of agitation could show dementia or cognitive impairments. For example, Elderly individuals and dementia
patients commonly experience disrupted sleep-wake cycles, which may lead to psychomotor agitation, confusion, and wandering.
Sundowning syndrome, which encompasses many of these behaviours, is characterized by a temporal pattern in the severity of symptoms,
usually expressed as worse during the late afternoon or evening. This could easily be observed in motor activity level and circadian rhythms. \cite{bedrosian2013sundowning}  
However, the research on this subject is quite limited to research on animals and there are few examples of this syndrome monitoring on real patients. 
A couple of solutions based on motion trackers were proposed. Aran and others have adopted a cross-entropy metrics method to evaluate the capability of
a model of predicting behavioural changes, defining an abstract layer to create a common ground for different sensor configurations. \cite{aran2016anomaly} This approach has
inspired the method used by ENRICHME project. 










